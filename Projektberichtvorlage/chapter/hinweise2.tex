%Arik Messerman, September 2006, uidarikmesserman301174 003 0021
%%% KAPITEL 3 ***
\addcontentsline{toc}{chapter}{Hinweise 2}
\chapter*{Hinweise 2}

\section{Test des Layouts} \label{kap3}
Um die Lesbarkeit zu erleichtern ist es ratsam jedes Kapitel mit Worten einzuleiten, die beschreiben was man gleich lesen wird. Einleitende Worte und ein kurzer Abriss worauf man mit der Wahl der Unterkapitel abzielt \dots

\section{Erste Betrachtungen}
Bevor wir uns mit Details besch�ftigen erstmal \dots
\subsection{Ganzheitliche Analyse}
Am Tag nach der unerwarteten Arbeitszeitverl�ngerung verbarg Stevens seine pers�nlichen Gef�hle. Fragen zum Kampf um seinen Arbeitsplatz bei Hertha BSC wollte der Niederl�nder nicht mehr h�ren. ,,�ber meine Person spreche ich nicht mehr, da ist alles gesagt worden'', sagte Stevens vier Tage vor dem ersten seiner beiden pers�nlichen Entscheidungsspiele gegen Hansa Rostock. Nur mit Siegen am Samstag in der Bundesliga und am kommenden Dienstag im DFB-Pokal kann der 49 Jahre alte Trainer seinen Job in Berlin retten.\\

Unter einem grauen Berliner Himmel schien nach dem Ultimatum vom Montag der Alltag auf dem Trainingsplatz eingekehrt zu sein. Erst nach der ersten von zwei �bungseinheiten machte die Zur�ckhaltung der Hertha-Profis den Ernst der Lage beim �berraschenden Tabellenletzten wieder klar. ,,Wir wollen diese Woche in Ruhe arbeiten. Das hat die Mannschaft so vereinbart, das ist keine Anweisung von oben'', betonte Nationalspieler Marko Rehmer im Namen seiner Kollegen. Der Mannschaftsrat hatte in einer Erkl�rung mitgeteilt, er trage die ungew�hnliche Vereinbarung mit Stevens mit. \\

Die �ffentlichkeit in der Hauptstadt reagierte auf die vorl�ufige Weiterbesch�ftigung von Stevens mit Unverst�ndnis. ,,Stevens f�r Stevens'', wunderte sich die ,,Berliner Zeitung'' am Dienstag. ,,Berlin unter Schock'', titelte die ,,BZ'', der ,,Berliner Kurier'' formulierte drastisch Richtung Hertha: ,,Hertha BSE - ihr seid doch irre!'' Die ,,Berliner Morgenpost'' schrieb: ,,Heldenmut oder Starrsinn -- Dieter Hoene� unter Druck.'' Und der ,,Tagesspiegel'' sprach von einer Stevens-Bilanz ,,schwach wie im Aufstiegsjahr''. Die ,,Galgenfrist'' (,,M�rkische Oderzeitung'') wertete ,,Bild'' als Votum gegen die Fan-Mehrheit. 


\subsection{Detaillierte Betrachtung des Sachverhalts}
Am Tag nach der unerwarteten Arbeitszeitverl�ngerung verbarg Stevens seine pers�nlichen Gef�hle. Fragen zum Kampf um seinen Arbeitsplatz bei Hertha BSC wollte der Niederl�nder nicht mehr h�ren. ,,�ber meine Person spreche ich nicht mehr, da ist alles gesagt worden'', sagte Stevens vier Tage vor dem ersten seiner beiden pers�nlichen Entscheidungsspiele gegen Hansa Rostock. Nur mit Siegen am Samstag in der Bundesliga und am kommenden Dienstag im DFB-Pokal kann der 49 Jahre alte Trainer seinen Job in Berlin retten.\\

Die �ffentlichkeit in der Hauptstadt reagierte auf die vorl�ufige Weiterbesch�ftigung von Stevens mit Unverst�ndnis. ,,Stevens f�r Stevens'', wunderte sich die ,,Berliner Zeitung'' am Dienstag. ,,Berlin unter Schock'', titelte die ,,BZ'', der ,,Berliner Kurier'' formulierte drastisch Richtung Hertha: ,,Hertha BSE - ihr seid doch irre!'' Die ,,Berliner Morgenpost'' schrieb: ,,Heldenmut oder Starrsinn -- Dieter Hoene� unter Druck.'' Und der ,,Tagesspiegel'' sprach von einer Stevens-Bilanz ,,schwach wie im Aufstiegsjahr''. Die ,,Galgenfrist'' (,,M�rkische Oderzeitung'') wertete ,,Bild'' als Votum gegen die Fan-Mehrheit. 

